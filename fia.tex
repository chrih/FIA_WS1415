\documentclass[a4paper,10pt]{scrreprt}
%\documentclass[a4paper,10pt]{scrartcl}

\usepackage[utf8]{inputenc}
\usepackage[german]{babel}
\usepackage[pdftex]{graphicx}
\usepackage{listings}
\usepackage{color}
\usepackage{amssymb}
\usepackage{marvosym}
\usepackage{amsmath}
\usepackage{array}
\usepackage{geometry}
\usepackage{color}

\newcommand{\f}{\noindent\textbf}
\geometry{verbose,tmargin=2cm,bmargin=2.5cm,lmargin=2.5cm,rmargin=3cm,headheight=80pt}

\title{FIA - WS 2014/15}
\author{}
\date{}

\begin{document}
\maketitle

\chapter{Introduction}
\section{Internet}

\begin{itemize}
 \item loosely hierarchical: tier 1 to tier 3
 \item communications infrastructure enables distributed applications
 \item hyper... $\Rightarrow$ concentration of content (Amazon, Google, ...)
\end{itemize}

\section{Protocol Layering}
\begin{itemize}
 \item ISO/OSI
 \item TCP, UDP, ICMP, UDP Like, SCTP, ...
 \item DHCP, NAT, ...
\end{itemize}

\section{Analogue digital conversion}
\begin{itemize}
 \item Fourier transformation 
 \item sampling theorem + Nyquist
 \item PCM-transmission 
 \item small amplitudes are being encoded more detailed than large amplitudes
\end{itemize}

\section{Color Coding}
\begin{itemize}
 \item monochrome, RGB, YCbCr
 \item RGB: 
 \begin{itemize}
  \item red, green, blue values between [0,  \{7,31,...,65535\}]
  \item nonlinear encoding of intensities 
  \item computer graphics
 \end{itemize}
 \item YCbCr
 \begin{itemize}
  \item TV and digital video
  \item Y more important $\Rightarrow$ encode with more details 
  \item more efficient than RGB
  \item can handle downsampling better
  \item YUV is based color model used in analog color TV
  \begin{itemize}
   \item YCbCr is scaled and offset version
   \item YPbPr is the analog version of YCbCr
  \end{itemize}
 \end{itemize}
\end{itemize}

\chapter{Digital coding of audio and video}
\section{Rate-Distortion Theory}
\begin{itemize}
 \item lossless compression algorithms 
 \begin{itemize}
  \item allow perfect reconstruction 
  \item low compression ratios
  \item frequently encountered data is encoded more efficiently 
 \end{itemize}
 \item lossy compression algorithms
 \begin{itemize}
  \item result is only a close approximation of original data
  \item trade-off: distortion vs. required rate 
  \item much higher compression rate than lossless compression
 \end{itemize}
//TODO: bild einfuegen
 \item rate and distortion as measures for efficiency of compression and difference between reconstructed and original data 
 \begin{itemize}
  \item goal is to minimize distortion and rate 
  \item basic problem:
  \begin{itemize}
   \item minimum expected distortion at a given rate?
   \item minimum rate to achieve given distortion?  
  \end{itemize}
 \end{itemize}
 \item distortion measures
 \begin{itemize}
  \item mean square error $\sigma^2 = \frac{1}{N}\sum\limits_{i=1}^N(x_i - y_i)^2$
  \item signal to noise ratio $SNR = 10 log_{10}\frac{\sigma_x^2}{\sigma_d^2}$
  \item peak signal to noise ratio $PSNR = 10 log_{10}\frac{x_{peak}^2}{x_d^2}$
 \end{itemize}
 \item in order to maximize efficient communication maximize mutual information between x and y 
 \item rate distortion function 
\end{itemize}

\section{digital image}
\begin{itemize}
 \item conversion between RGB and YUV
 \item downsampling J:a:b
 \begin{itemize}
  \item 4:4:4 $\widehat{=}$ no downsampling
  \item 4:2:2 $\widehat{=}$ 
  \item 4:2:0 $\widehat{=}$ 
 \end{itemize}
 \item statistical image modeling 
 \begin{itemize}
  \item all natural images occupy a tiny and unknown sloped space of all images 
  \item pixel intensities are dependent and correlated to the corresponding image $\Rightarrow$ pixel within images are highly correlated 
  \item correlation of pixels has high impact on image compression
 \end{itemize}
 \item image transformations 
 \begin{itemize}
  \item negative transformations
  \item log transformations
  \item power-law transformations
 \end{itemize}
 \item intensity:
 \begin{itemize}
  \item change in brightness $\Rightarrow$ shift of histogram
  \item change in contrast $\Rightarrow$ stretch/? of histogram
 \end{itemize}
\item filters based on convolution of neighbor pixels $\Rightarrow$ enhancement, smoothing, edge, detection, ...
\item the hoar transform 
\begin{itemize}
 \item simplest useful energy compression
 \item lossless retransformation is possible 
 \item transform original image into 4 parts (in 20)
 \begin{enumerate}
  \item top-left: a+b+c+d, 4 point average, very important
  \item top-right: a-b+c-d, average horizontal gradient, less important
  \item bottom-left: a+b-c-d, average vertical gradient, less important
  \item bottom-right: a-b-c+d, diagonal curvature, less important
 \end{enumerate}
 \item 1 is more expensive while 2-4 is cheaper to encode 
 \item reordering required to provide ``typical'' representation
\end{itemize}
 \item entropy -- quantifying the required bitrate
 \begin{itemize}
  \item entropy $H_x$ represents the mean number of bits per pixel that are required to encode the image 
  \item $H_x = \sum\limits_{i=0}^{M-1} p_i-log_2\left(\frac{1}{p_i}\right)$ where $p_i = \frac{\text{pixel in bin i}}{N}$ where N is the number of pixels
  \item estimated number of bits needed is $H_x\cdot N$
 \end{itemize}
 \item the Karkunen-Loeve transform (KLT)
 \begin{itemize}
  \item is optimal for minimizing bitrate
  \item not or seldom used in practice $\Rightarrow$ slow and complex
 \end{itemize}

\end{itemize}

\end{document}
